Title: "Test of Lepton Universality in b→sℓ+ℓ− Decays"

Abstract: A brief summary of the paper's main findings and their significance. Mention that this is the first simultaneous test of muon-electron universality using B+ → K+ℓ+ℓ− and B0 → K∗0ℓ+ℓ− decays, and that the results are compatible with the predictions of the Standard Model.

Introduction:
    Background on the topic of lepton universality.
    Importance and implications of testing lepton universality.
    Briefly mention previous relevant studies and their findings.
    State the objective of the current study.

Methodology:
    Description of the LHCb detector and the data collection process between 2011 and 2018.
    Explanation of the use of beauty mesons produced in proton-proton collisions.
    Description of the analysis method, including the use of the double ratio and the simultaneous maximum-likelihood fit.

Results:
    Presentation of the main findings, including the compatibility of the results with the predictions of the Standard Model.
    Discussion on the impact of the calibration chain on the double ratio.
    Mention of the reduction in systematic uncertainties.

Discussion:
    Interpretation of the results and their implications for the field.
    Comparison of the findings with those of previous studies.
    Discussion of the strengths and limitations of the study.

Conclusion:
    Summary of the key findings and their implications.
    Potential directions for future research.

Acknowledgements: Acknowledge any funding sources, contributions from colleagues or institutions, etc.

References: Cite all the sources of information used in the paper.