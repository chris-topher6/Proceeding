The Standard Model of particle physics, a cornerstone in our 
understanding of the fundamental constituents of the universe, 
has been remarkably successful in explaining the behavior and 
interactions of elementary particles. Despite its success, it 
leaves several phenomena unexplained, such as the nature of dark 
matter, the matter-antimatter asymmetry in the universe, and the 
integration of gravity at the quantum level. These gaps in our 
understanding motivate the search for Physics beyond the Standard 
Model (BSM).

One of the key features of the Standard Model is the principle of 
lepton universality (LFU), which posits that all leptons ($e$, $\mu$ and $\tau$) 
should interact in the same way with gauge bosons, apart from differences 
due to their distinct masses. 
Although the leptonflavour in the standard model is not yet theoretically 
understood as a conservation quantity corresponding to a symmetry, 
it has been experimentally confirmed with high precision that W and Z 
bosons couple equally to different leptonflavours \cite{LU_CDF}, \cite{LU_ATLAS}.
Any violation of this principle could be 
a signal of new physics, providing a window into BSM Physics.

In this context, $b\to sl^+l^-$ decays in not resonant regions, provide a 
fertile testing ground, as these decays are sensitive to the Wilson 
coefficients $C_7^{(')}$ corresponding to the electromagnatic dipole operator, 
$C_{9l}^{(')}$ corresponding to the vector operator, and $C_{10,l}^{(')}$ corresponding 
to the axialvector operator. Consequently, LFU measurements can potentially detect 
the influence of hypothetical heavy particles. %Quelle?
The LHCb experiment at CERN, designed to study the 
properties of particles containing b (beauty) and c (charm) quarks, offers 
a unique opportunity to probe these decays with unprecedented precision.