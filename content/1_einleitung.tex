The Standard Model of particle physics (SM), a cornerstone of our 
understanding of the fundamental constituents of the universe, 
has been remarkably successful in explaining the behavior 
and interactions of elementary particles. Despite its success, 
it leaves several phenomena unexplained, such as the 
nature of dark matter, the matter-antimatter asymmetry in the 
universe, and the integration of gravity at the quantum level. 
These gaps in our understanding motivate the search for Physics 
Beyond the Standard Model (BSM).\cite{SM_BSM}

A key feature of the SM is the principle of Lepton 
Flavor Universality (LFU), which posits that all leptons ($e$, 
$\mu$, and $\tau$) should interact identically with gauge bosons, 
apart from differences due to their distinct masses. Although 
lepton flavor in the SM is not theoretically understood 
as a conservation quantity corresponding to a symmetry, it has 
been experimentally confirmed with high precision that $W$ and $Z$ 
bosons couple equally to different lepton flavors \cite{LU_CDF}, 
\cite{LU_ATLAS}. Any violation of this principle could signal new 
physics, providing a window into BSM Physics.

In this context, non-resonant $b\to s\ell^+\ell^-$ decays provide a fertile 
testing ground. These decays are flavor-changing neutral currents, 
forbidden at the tree level in the SM, leading to high suppression. 
This makes them sensitive probes for BSM physics. Specifically, 
$b\to s\ell^+\ell^-$ decays are sensitive to the Wilson coefficients 
$C_7^{\scriptscriptstyle (')}$, $C_{9l}^{\scriptscriptstyle (')}$, and 
$C_{10,l}^{\scriptscriptstyle (')}$, corresponding to the electromagnetic 
dipole, vector, and axial vector operators. 
Consequently, LFU measurements can potentially detect the influence 
of hypothetical heavy particles.

