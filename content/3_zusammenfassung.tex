The measurements of $R_K$ and $R_K^*$ in the low $q^2$ 
and central $q^2$ regions currently represent the most 
precise tests of Lepton Flavour Universality (LFU). 
Contrary to previous measurements, this analysis aligns 
with the Standard Model predictions, with an agreement of 
$\num{0.2}\sigma$, as shwon in figure \ref{fig:results}. 
This study underscores the importance of understanding 
electron misidentification and also demonstrates the 
benefits of a double-ratio approach in minimizing 
systematic uncertainties. The statistical uncertainties 
significantly outweigh the systematic ones in this 
analysis, suggesting that a larger dataset is required 
to improve the precision of the measurement.

\begin{figure}
    \centering
    \includegraphics[width=\linewidth]{figures/results.png}
    \caption{Graphical comparison of the measured results of $R_K$ and $R_{K^*}$ in the low and central $q^2$ range with the standard model predictions \cite{lhcbcollaboration2022measurement}.}
    \label{fig:results}
\end{figure}

While the LFU tensions with respect to the Standard Model 
appear to be of systematic origin, the tensions related to 
the branching fractions $\mathcal{B}(B^{(0,+)}\to K^{(+,*0)}l^+l^-)$ 
remain \cite{Branchingfraction}.

The upcoming LHCb Run 3, with its increased statistics and 
new detector enhancements, including the trigger system, is 
expected to further refine the precision of LFU tests and 
potentially provide clarity on all $B$ anomalies.